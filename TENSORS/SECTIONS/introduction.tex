\section{Introduzione}

\noindent I metodi di analisi lineare su tensori rappresentano un ambito di ricerca relativamente recente \cite{HOSVD}. Per questa ragione, il loro utilizzo in campo combustivo è ancora limitato. Tuttavia, i dati sperimentali e numerici provenienti da simulazioni sono frequentemente disponibili in forma tensoriale, rendendo lo studio di questi metodi particolarmente promettente per l'analisi di flussi reattivi.

\noindent Queste note presentano una discussione intuitiva del significato fisico delle componenti di tali metodi quando applicati a dati combustivi. La trattazione matematica rigorosa è disponibile principalmente in \cite{HOSVD} e \cite{HOSVD1}.

\noindent Esistono diverse strategie per decomporre un tensore, ciascuna con caratteristiche e applicazioni specifiche. La scelta del metodo dipende dalle proprietà dei dati e dagli obiettivi dell'analisi.

\subsection{Rango tensoriale e metodologie di decomposizione}

\noindent Il concetto di rango per un tensore è più complesso rispetto al caso matriciale. Per un tensore di ordine $N$, si possono definire diversi tipi di rango:

\begin{itemize}
    \item \textbf{Rango CP (CANDECOMP/PARAFAC)}: il numero minimo di tensori di rango 1 la cui somma ricostruisce il tensore originale
    \item \textbf{Rango Tucker}: una generalizzazione del rango matriciale, definito per ciascuna modalità del tensore
    \item \textbf{Rango multilineare}: legato alla dimensione del tensore core nella decomposizione di Tucker
\end{itemize}
\begin{comment}

Le principali metodologie di decomposizione tensoriale includono:

\textbf{Decomposizione CP}: esprime il tensore come somma di tensori di rango 1, offrendo interpretabilità ma potendo risultare numericamente instabile.

\textbf{Decomposizione di Tucker (HOSVD)}: generalizza la SVD matriciale tramite un tensore core e matrici fattoriali per ciascuna modalità, fornendo una rappresentazione più flessibile e stabile.

\textbf{Decomposizioni tensoriali train}: utilizzate per tensori di ordine elevato, offrono vantaggi computazionali in termini di storage e operazioni.
\end{comment}

\noindent Il main focus per ora e' HOSVD perche' e' quello che si collega meglio a i metodi gia' presenti nel campo.